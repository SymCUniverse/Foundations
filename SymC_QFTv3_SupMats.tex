\documentclass[11pt]{article}

\usepackage[top=0.50in,bottom=0.75in,left=0.75in,right=0.75in]{geometry}
\usepackage{amsmath,amssymb}
\usepackage{graphicx}
\usepackage{hyperref}
\usepackage{bm}
\usepackage{braket}
\usepackage{physics}
\usepackage{siunitx}
\usepackage{booktabs}
\usepackage[numbers,sort&compress]{natbib}

\title{Supplemental Material:\\Structural Constraints from Critical Damping in Open Quantum Field Theories: Implications for QCD Substrate Inheritance and Phenomenological Extensions}
\author{Nate Christensen\\
SymC Universe Project, Missouri, USA\\
\href{mailto:NateChristensen@SymCUniverse.com}{NateChristensen@SymCUniverse.com}}
\date{February 3, 2026}

\begin{document}

\maketitle

\section{Detailed Schwinger-Keldysh Derivation}
\label{sec:sk_details}

\subsection{Influence Functional Calculation}

The microscopic action for a scalar field $\phi$ coupled to a bath of oscillators $\{q_\alpha\}$ is:
\begin{equation}
S[\phi,\{q_\alpha\}] = \int d^4x \left[ \frac{1}{2}(\partial\phi)^2 - \frac{1}{2}m^2\phi^2 \right] + \sum_\alpha \int d^4x \left[ \frac{1}{2}(\partial q_\alpha)^2 - \frac{1}{2}\omega_\alpha^2 q_\alpha^2 - g_\alpha \phi q_\alpha \right].
\end{equation}

On the Keldysh contour with forward ($+$) and backward ($-$) branches:
\begin{equation}
Z = \int \mathcal{D}\phi_\pm \mathcal{D}q_{\alpha,\pm} \exp\left[i(S[\phi_+,q_{+}] - S[\phi_-,q_{-}])\right].
\end{equation}

Integrating out the bath oscillators yields:
\begin{equation}
\exp[i S_{\text{IF}}[\phi_+,\phi_-]] = \int \mathcal{D}q_{\alpha,\pm} \exp\left[i\sum_\alpha(S_{\alpha}[q_{\alpha,+}] - S_{\alpha}[q_{\alpha,-}] - \int d^4x g_\alpha(q_{\alpha,+}\phi_+ - q_{\alpha,-}\phi_-))\right].
\end{equation}

The resulting influence functional in Keldysh space:
\begin{equation}
S_{\text{IF}} = \frac{1}{2}\int d^4x d^4x' [\phi_+(x)-\phi_-(x)]\Sigma^{+-}(x-x')[\phi_+(x')+\phi_-(x')] + \text{(Keldysh/advanced terms)}.
\end{equation}

The retarded self-energy:
\begin{equation}
\Sigma_R(x-x') = \Sigma^{++}(x-x') - \Sigma^{+-}(x-x') = \sum_\alpha g_\alpha^2 G_{R,\alpha}(x-x'),
\end{equation}
where $G_{R,\alpha}$ is the retarded Green's function for bath oscillator $\alpha$.

\subsection{Markovian Limit Derivation}

For a bath with spectral density $J(\omega) = \sum_\alpha g_\alpha^2 \delta(\omega-\omega_\alpha)$, the retarded self-energy in frequency space:
\begin{equation}
\Sigma_R(\omega) = \int_0^\infty d\omega' \frac{J(\omega')}{\omega - \omega' + i0^+}.
\end{equation}

For Ohmic dissipation $J(\omega) = \gamma \omega$, and assuming $\omega \ll \omega_c$ (cutoff):
\begin{equation}
\Sigma_R(\omega) \approx -i\gamma\omega + \mathcal{O}(\omega^2/\omega_c).
\end{equation}

This yields the effective equation of motion:
\begin{equation}
(\square + m^2)\phi(x) + \gamma(u\cdot\partial)\phi(x) = \xi(x),
\end{equation}
where $\xi(x)$ represents quantum and thermal noise. The retarded response equation (mean field) follows by setting $\langle \xi \rangle = 0$:
\begin{equation}
(\square + m^2)\phi(x) + \gamma(u\cdot\partial)\phi(x) = 0.
\end{equation}

The Markovian approximation requires bath correlation time $\tau_{\text{bath}} \sim 1/\omega_c \ll 1/\omega_{\text{system}}$, justifying the local-in-time damping term.

\section{Fisher Information Rate Derivation}
\label{sec:fisher_derivation}

\subsection{Measurement Model}

Consider continuous quadrature measurement of a damped oscillator with equation of motion:
\begin{equation}
\ddot{x} + \gamma\dot{x} + \omega^2 x = A\cos(\omega_d t) + \xi(t),
\end{equation}
where $A$ is the unknown drive amplitude to be estimated, and measurement record:
\begin{equation}
y(t) = x(t) + \eta(t),
\end{equation}
with Gaussian white noise $\langle \eta(t)\eta(t')\rangle = \sigma^2\delta(t-t')$.

\subsection{Fisher Information Calculation}

The steady-state response to drive is:
\begin{equation}
x_{\text{ss}}(t) = \frac{A}{\sqrt{(\omega^2-\omega_d^2)^2 + \gamma^2\omega_d^2}}\cos(\omega_d t + \phi).
\end{equation}

For on-resonance driving ($\omega_d = \omega$):
\begin{equation}
x_{\text{ss}}(t) = \frac{A}{\gamma\omega}\cos(\omega t).
\end{equation}

The Fisher information for parameter $A$ after time $T$:
\begin{equation}
\mathcal{I}(A) = \frac{1}{\sigma^2}\int_0^T dt \left(\frac{\partial x_{\text{ss}}}{\partial A}\right)^2 = \frac{T}{2\sigma^2} \left(\frac{1}{\gamma\omega}\right)^2.
\end{equation}

The information rate:
\begin{equation}
I = \lim_{T\to\infty}\frac{\mathcal{I}(A)}{T} = \frac{1}{2\sigma^2\gamma^2\omega^2} = \frac{1}{2\sigma^2}\frac{1}{(\gamma\omega)^2}.
\end{equation}

In dimensionless form with $\chi = \gamma/(2\omega)$:
\begin{equation}
I(\chi) = \frac{1}{8\sigma^2\omega^2\chi^2} \propto \frac{1}{\chi^2}.
\end{equation}

However, the oscillator response bandwidth scales as $\Delta\omega \sim \gamma$, reducing effective information at large $\chi$. The correct form accounting for bandwidth:
\begin{equation}
I(\chi) = \frac{|\omega|}{\gamma}\frac{1}{1+(\gamma/2|\omega|)^2} = \frac{1}{2\chi(1+\chi^2)}.
\end{equation}

\subsection{Entropy Production Rate}

The entropy production from coupling to a thermal bath at temperature $T$:
\begin{equation}
\Sigma = \gamma\int_0^\infty dt \langle \dot{x}^2(t)\rangle \coth\left(\frac{\omega}{2T}\right).
\end{equation}

In high-temperature limit $T \gg \omega$:
\begin{equation}
\Sigma \approx 2\gamma T\langle \dot{x}^2\rangle_{\text{steady-state}}.
\end{equation}

For driven steady-state:
\begin{equation}
\langle \dot{x}^2\rangle = \frac{A^2\omega^2}{2\gamma^2\omega^2} = \frac{A^2}{2\gamma^2}.
\end{equation}

Thus:
\begin{equation}
\Sigma \propto \gamma T.
\end{equation}

The information efficiency:
\begin{equation}
\eta(\chi) = \frac{I}{\Sigma} \propto \frac{1}{\chi(1+\chi^2)}.
\end{equation}

Optimization yields $d\eta/d\chi = 0$ at $\chi=1$.

\section{DUNE Spectral Data Tables}
\label{sec:dune_data}

\subsection{Appearance Probability Predictions}

Table \ref{tab:dune_data} provides quantitative predictions for $\nu_\mu \to \nu_e$ appearance probability at DUNE baseline $L = 1300$ km with matter density profile averaged along the path.

\begin{table}[h]
\centering
\caption{DUNE spectral tilt predictions. Standard PMNS uses normal ordering with $\sin^2\theta_{13} = 0.022$, $\sin^2\theta_{23} = 0.5$, $\Delta m_{31}^2 = 2.5 \times 10^{-3}$ eV$^2$. Damping model uses effective rate $\Gamma_{\text{eff}} = 3 \times 10^{-23}$ eV integrated over baseline. Probabilities are shown in arbitrary units for compact display. Relative suppression is defined as $S(E) \equiv 1 - P_{\text{SymC}}/P_{\text{PMNS}}$.}
\label{tab:dune_data}
\begin{tabular}{cccc}
\toprule
Energy $E$ (GeV) & $P_{\text{PMNS}}$ & $P_{\text{SymC}}$ & Relative Suppression $S(E)$ \\
\midrule
1.0 & 8.50 & 8.49 & $1.18 \times 10^{-3}$ \\
1.5 & 4.20 & 4.19 & $2.38 \times 10^{-3}$ \\
2.0 & 2.30 & 2.29 & $4.35 \times 10^{-3}$ \\
3.0 & 0.95 & 0.91 & $4.21 \times 10^{-2}$ \\
5.0 & 0.35 & 0.32 & $8.57 \times 10^{-2}$ \\
\bottomrule
\end{tabular}
\end{table}


\subsection{Tilt Parameter Extraction}

The observable tilt parameter compares suppression at high versus low energy:
\begin{equation}
\alpha = \frac{(P_{\text{PMNS}} - P_{\text{SymC}})_{E=3\text{ GeV}}}{(P_{\text{PMNS}} - P_{\text{SymC}})_{E=1\text{ GeV}}}.
\end{equation}

From Table \ref{tab:dune_data}:
\begin{align}
\text{Suppression at } E &= 3 \text{ GeV}: \quad 0.95 - 0.91 = 0.04, \\
\text{Suppression at } E &= 1 \text{ GeV}: \quad 8.50 - 8.49 = 0.01, \\
\alpha &= \frac{0.04}{0.01} = 4.0.
\end{align}


This exceeds the main text conservative estimate $\alpha = 1.15 \pm 0.05$ due to the exponential damping factor $\exp(-\Gamma L/E)$. The energy scaling $\Gamma \propto 1/E$ in the matter-dependent damping model produces enhanced suppression at lower energies, opposite to standard MSW resonance effects.

\subsection{No-Global-Critical Hierarchy Data}

Table \ref{tab:hierarchy_data} quantifies the mass-ordered damping hierarchy across matter density scales.

\begin{table}[h]
\centering
\caption{Damping ratio hierarchy for neutrino mass eigenstates. Values computed using density-dependent damping law Eq. (37) with normal mass ordering $m_1 < m_2 < m_3$ and energy $E = 2$ GeV representative of atmospheric and accelerator neutrinos.}
\label{tab:hierarchy_data}
\begin{tabular}{ccccc}
\toprule
Density $\rho/\rho_{\text{Earth}}$ & $\chi_1$ & $\chi_2$ & $\chi_3$ & Stability Status \\
\midrule
$1$ (Terrestrial) & 0.135 & 0.004 & 0.0001 & All underdamped \\
$10^2$ & 0.850 & 0.025 & 0.0006 & Approaching EP \\
$2 \times 10^2$ & 1.200 & 0.035 & 0.0009 & $\chi_1$ overdamped \\
$10^4$ & 6.030 & 0.177 & 0.0044 & Persistent beat \\
$10^6$ (Neutron star) & 60.30 & 1.770 & 0.0440 & $\chi_2$ overdamped \\
\bottomrule
\end{tabular}
\end{table}

\textbf{Key observation}: Even at neutron star core densities ($\rho \sim 10^6 \rho_{\text{Earth}} \approx 10^{14}$ g/cm$^3$), the heaviest mass eigenstate maintains $\chi_3 \ll 1$, preserving high-frequency oscillatory structure. This No-Global-Critical constraint is the most stringent falsification test: simultaneous transition $\chi_1 = \chi_2 = \chi_3 = 1$ at any density-energy combination would refute the framework.

\section{Lattice QCD Calculation Details}
\label{sec:lattice_details}

\subsection{Glueball Spectral Function}

The $0^{++}$ glueball two-point correlator:
\begin{equation}
G(\tau) = \int d^3x \langle \mathcal{O}(x,\tau)\mathcal{O}(0,0)\rangle,
\end{equation}
where $\mathcal{O} = \text{Tr}(F_{\mu\nu}F^{\mu\nu})$ is the scalar glueball operator.

Spectral representation:
\begin{equation}
G(\tau) = \int_0^\infty d\omega A(\omega) K(\omega,\tau),
\end{equation}
with thermal kernel:
\begin{equation}
K(\omega,\tau) = \frac{\cosh[\omega(\tau-1/(2T))]}{\sinh[\omega/(2T)]}.
\end{equation}

At zero temperature ($T \to 0$):
\begin{equation}
K(\omega,\tau) = e^{-\omega\tau}.
\end{equation}

Maximum entropy method reconstructs $A(\omega)$ by maximizing:
\begin{equation}
S[A] = -\int d\omega A(\omega)\ln\left[\frac{A(\omega)}{m(\omega)}\right] + \alpha\chi^2[A],
\end{equation}
where $m(\omega)$ is default model and $\chi^2$ measures fit to data $G(\tau)$.

\subsection{Critical Damping Extraction}

From reconstructed $A(\omega)$, fit to Breit-Wigner form near lowest resonance:
\begin{equation}
A(\omega) = \frac{\Gamma m}{\pi[(m^2-\omega^2)^2 + m^2\Gamma^2]}.
\end{equation}

Extract $m$ and $\Gamma$, then compute:
\begin{equation}
\chi_{\text{QCD}} = \frac{\Gamma}{2m}.
\end{equation}

Current lattice estimates: $m_{0^{++}} \approx 1.7$ GeV, $\Gamma_{0^{++}} \sim 0.2$ to $0.4$ GeV (large uncertainties), giving $\chi \sim 0.06$ to $0.12$. This is far from critical damping, indicating either:
\begin{enumerate}
\item The $0^{++}$ glueball is not the relevant substrate
\item Additional QCD dynamics modify the effective damping
\item The substrate inheritance mechanism requires refinement
\end{enumerate}

This represents a current challenge to the framework requiring resolution.

\section{Cosmological Perturbation Details}
\label{sec:cosmo_details}

\subsection{Growth Factor Equation}

In synchronous gauge, the density contrast $\delta = \delta\rho_m/\rho_m$ evolves as:
\begin{equation}
\ddot{\delta} + 2H\dot{\delta} - \frac{3}{2}H^2\Omega_m(a)\delta = 0,
\end{equation}
where $H = \dot{a}/a$ and $\Omega_m(a) = \Omega_{m,0}a^{-3}/(E(a))^2$ with:
\begin{equation}
E(a) = \sqrt{\Omega_{m,0}a^{-3} + \Omega_{\Lambda,0}}.
\end{equation}

Define effective frequency and damping:
\begin{align}
\omega_\delta^2(a) &= \frac{3}{2}H^2\Omega_m(a), \\
\gamma_\delta(a) &= 2H.
\end{align}

The critical damping ratio:
\begin{equation}
\chi_\delta(a) = \frac{\gamma_\delta}{2\omega_\delta} = \frac{2H}{2\sqrt{(3/2)H^2\Omega_m}} = \sqrt{\frac{2}{3\Omega_m(a)}}.
\end{equation}

At $\chi_\delta = 1$:
\begin{equation}
\Omega_m(a_*) = \frac{2}{3}.
\end{equation}

Using flatness $\Omega_m + \Omega_\Lambda = 1$:
\begin{equation}
\Omega_\Lambda(a_*) = \frac{1}{3}.
\end{equation}

The deceleration parameter:
\begin{equation}
q = -\frac{\ddot{a}a}{\dot{a}^2} = \frac{\Omega_m}{2} - \Omega_\Lambda.
\end{equation}

At transition:
\begin{equation}
q(a_*) = \frac{2/3}{2} - \frac{1}{3} = \frac{1}{3} - \frac{1}{3} = 0.
\end{equation}

This completes the proof of $\chi_\delta = 1 \iff q = 0$.

\bibliographystyle{apsrev4-1}
\begin{thebibliography}{99}
\bibitem{Keldysh1965}
L. V. Keldysh, ``Diagram technique for nonequilibrium processes,'' \textit{Sov. Phys. JETP} \textbf{20}, 1018 (1965).

\bibitem{Sieberer2016}
L. M. Sieberer, M. Buchhold, and S. Diehl, ``Keldysh field theory for driven open quantum systems,'' \textit{Rep. Prog. Phys.} \textbf{79}, 096001 (2016).

\bibitem{Clerk2010}
A. A. Clerk et al., ``Introduction to quantum noise, measurement, and amplification,'' \textit{Rev. Mod. Phys.} \textbf{82}, 1155 (2010).
\end{thebibliography}

\end{document}

\documentclass[12pt]{article}

\usepackage{amsmath, amssymb, bm, physics}
\usepackage{geometry}
\usepackage{hyperref}
\usepackage{graphicx}

\geometry{top=0.5in, bottom=0.75in, left=0.75in, right=0.75in}

\title{SymC and the QFT:\\
Critical-Damping Boundary Dynamics of Dissipative Quantum Fields\\v2}

\author{Nate Christensen\\[4pt]
\small Symmetry Universe Project, Missouri, USA}

\date{17 November 2025
}

\begin{document}
\maketitle

\begin{abstract}
Version 2 clarifies the role of the $\chi = 1$ boundary, adds the information-efficiency interpretation, incorporates the inheritance mechanism as a structural constraint, and removes material not directly related to dissipative quantum fields.

Symmetrical Convergence (SymC) identifies the dimensionless ratio
$\chi \equiv \gamma/(2|\omega|)$ as a structural boundary separating oscillatory
and monotone dynamics in open quantum systems. At $\chi = 1$, each field mode
reaches a non-Hermitian Exceptional Point (EP), where the retarded propagator's
poles coalesce and the causal response function becomes
$g(t) = \Theta(t)\, t e^{-|\omega| t}$.
This manuscript provides the complete open-QFT formulation of SymC, including
theoretical foundations, covariance, substrate inheritance, renormalization-group
stability, non-Markovian EP broadening, interpretation of the damping transition,
and quantitative laboratory falsification. The analysis emphasizes the structural
role of the $\chi=1$ boundary as both a dynamical separatrix and an
information-efficiency extremum, and shows that it remains robust under substrate
inheritance, weak renormalization-group flow, and controlled non-Markovian
broadening.
\end{abstract}

\section{Modeling Assumptions and Scope}
For a scalar field $\phi(x)$ weakly coupled to an environment, integrating out the
bath degrees of freedom through influence-functional techniques yields the effective
equation
\begin{equation}
(\Box + m^{2})\phi + \gamma (u \cdot \partial)\phi = 0,
\end{equation}
where $\gamma$ is a dissipative damping rate and $u^{\mu}$ the bath four-velocity.
This form applies in the regime of weak coupling, short memory, and negligible spatial
gradients in bath correlations. Closed quantum field theories without environmental
interaction are outside the scope of SymC. The present work treats open QFTs where
irreversible damping is physically realized. This framework builds upon the SymC
interpretation of the electromagnetic vacuum as an effectively lossless $\chi = 0$
photon substrate; the present analysis characterizes the $\chi = 1$ interaction
boundary that emerges when those fields couple to a dissipative environment.

\section{Field-Mode Dynamics and the SymC Discriminant}
Fourier-decompose the field
\begin{equation}
\phi(x) = \int \frac{d^{3}k}{(2\pi)^{3}}\, q_{k}(t)\, e^{i\mathbf{k}\cdot\mathbf{x}}.
\end{equation}
The mode equation of motion is
\begin{equation}
\ddot{q}_{k} + \gamma_{k} \dot{q}_{k} + \omega_{k}^{2} q_{k} = 0,
\qquad
\omega_{k}^{2} = k^{2} + m_{\mathrm{eff}}^{2}.
\end{equation}
Define the SymC ratio
\begin{equation}
\chi_{k} \equiv \frac{\gamma_{k}}{2|\omega_{k}|}.
\end{equation}
Then
\begin{itemize}
\item $\chi_{k} < 1$: underdamped, oscillatory;
\item $\chi_{k} = 1$: critical damping, EP;
\item $\chi_{k} > 1$: overdamped, monotone.
\end{itemize}
The discriminant $\Delta_{k} = \gamma_{k}^{2} - 4\omega_{k}^{2}$ encodes these regimes.

\section{Exceptional Point and Spectral Signature}
The retarded propagator is
\begin{equation}
G^{R}_{k}(\Omega) =
\frac{1}{-\Omega^{2} - i\gamma_{k}\Omega + \omega_{k}^{2}}.
\end{equation}
The poles are
\begin{equation}
\Omega_{\pm} = -\frac{i\gamma_{k}}{2}
\pm \sqrt{\omega_{k}^{2} - \frac{\gamma_{k}^{2}}{4}}.
\end{equation}
At $\chi_{k} = 1$ the poles coalesce at $\Omega = -i|\omega_{k}|$, producing the EP.
The corresponding causal kernel is
\begin{equation}
g_{k}(t) = \Theta(t)\, t e^{-|\omega_{k}| t}.
\end{equation}
This kernel has no oscillatory component, defines a single relaxation timescale, and
encodes the fastest non-oscillatory decay. The spectral function
\begin{equation}
A_{k}(\Omega) = \frac{\gamma_{k}\Omega}{\bigl(\omega_{k}^{2}-\Omega^{2}\bigr)^{2}
+ \gamma_{k}^{2}\Omega^{2}},
\end{equation}
exhibits two peaks for $\chi_{k} < 1$ and a merged zero-detuning peak for
$\chi_{k} \approx 1$.

\begin{figure}[t]
  \centering
  \includegraphics[width=0.7\textwidth]{fig1_pole_trajectories.pdf}
  \caption{Pole structure of the damped mode as the SymC ratio
  $\chi_k = \gamma_k/(2|\omega_k|)$ is varied. For $\chi_k < 1$ the retarded
  propagator has two complex-conjugate poles with nonzero real parts
  (underdamped regime). At $\chi_k = 1$ the poles coalesce at
  $\Omega = -i|\omega_k|$, reaching the $\chi_k = 1$ critical-damping
  point (an Exceptional Point, EP). For $\chi_k > 1$ both poles lie on the
  imaginary axis and dynamics become strictly monotone.}
  \label{fig:pole-trajectories}
\end{figure}

\begin{figure}[t]
  \centering
  \includegraphics[width=0.7\textwidth]{fig2_spectral_merger.pdf}
  \caption{Spectral function $A_k(\Omega)$ for a single mode at three values
  of the SymC ratio: $\chi_k = 0.5$ (underdamped doublet), $\chi_k = 1.0$
  (merged peak at zero detuning), and $\chi_k = 1.3$ (overdamped single peak).
  The zero-detuning peak merger at $\chi_k \approx 1$ provides the primary
  experimental falsifier.}
  \label{fig:spectral-merger}
\end{figure}

\section{Covariance of the SymC Condition}
Including a bath velocity $u^{\mu}$, the inverse propagator becomes
\begin{equation}
D(k) = -\bigl(k^{2} - m^{2}\bigr) - i\gamma (k\cdot u).
\end{equation}
Thus the EP condition is covariant:
\begin{equation}
\gamma = 2|k\cdot u|.
\end{equation}
For any fixed mode, the truth of this condition is frame-independent. However,
different inertial frames identify different sets of modes as critical due to
transformations of $k^{\mu}$.

\section{Information-Efficiency Extremum}
The $\chi = 1$ boundary is not just a dynamical feature but a thermodynamic optimum.
Within this setting, an information-efficiency functional is defined as
\begin{equation}
\eta(\chi) \equiv \frac{I(\chi)}{\Sigma(\chi)},
\end{equation}
representing the ratio of information throughput $I$ to entropy production $\Sigma$.
Within the broader SymC framework, this functional $\eta(\chi)$ attains a strict
local maximum at $\chi = 1$:
\begin{itemize}
\item For $\chi < 1$ (underdamped), the system ``rings,'' wasting energy in persistent
oscillations and corrupting information with overshoot and ringing artifacts.
\item For $\chi > 1$ (overdamped), the system is rigid and relaxes slowly, sacrificing
responsiveness and information throughput for excessive stability.
\end{itemize}
The $\chi = 1$ boundary is therefore the unique state of maximal efficiency, achieving
the fastest, most stable response for a given energetic cost. This provides a physical
\emph{purpose} for the boundary's selection: systems driven by dynamical and
thermodynamic pressures are steered toward this near-optimal operating point.

\section{Substrate Inheritance: Mechanism and Stability}

\subsection{Mechanism}
Within the broader SymC program, \emph{Substrate Inheritance} is the principle that
stable, complex systems at a macroscopic scale $L$ cannot be built from persistently
unstable, ``ringing'' substrates at the microscopic QFT scale $L-1$. Dynamical and
physical constraints naturally bias foundational QFT substrates toward stabilization
in the near-critical adaptive window ($0.8 \lesssim \chi \lesssim 1.0$). This damping
structure is then inherited by higher-level dynamical systems, regardless of physical
domain, whenever stability constraints impose analogous response requirements.

\subsection{Stability}
Substrate inheritance introduces shifts
\begin{equation}
\omega \rightarrow \omega + \delta\omega, \qquad
\gamma \rightarrow \gamma + \delta\gamma.
\end{equation}
Thus
\begin{equation}
\delta\chi = \chi\left(\frac{\delta\gamma}{\gamma} - \frac{\delta\omega}{\omega}\right).
\end{equation}
In inheritance models, the same structural features of the substrate typically
renormalize $\omega$ and $\gamma$ proportionally,
\begin{equation}
\frac{\delta\gamma}{\gamma} =
\frac{\delta\omega}{\omega} + \mathcal{O}(\varepsilon^{2}),
\end{equation}
so
\begin{equation}
\delta\chi = \mathcal{O}(\varepsilon^{2}).
\end{equation}
Thus $\chi = 1$ is a stable, self-consistent condition under substrate perturbations,
with deviations only entering at second order.

\section{Renormalization-Group Stability}
A toy RG model with quartic coupling $\lambda$ yields
\begin{equation}
\frac{d\chi}{d\ell} = -\frac{5\lambda}{64\pi^{2}}\,\chi.
\end{equation}
For weak coupling $\lambda \sim 0.1$, $\chi$ changes by less than $1\%$ across three
decades of scale. Thus $\chi = 1$ remains approximately invariant under scale evolution
and constitutes a robust near-fixed manifold in the weak-coupling regime.

\section{Non-Markovian Memory and the Near-Critical Band}
Let the memory kernel be
\begin{equation}
K(t) = \gamma_{0}e^{-\Gamma t}.
\end{equation}
Then
\begin{equation}
\gamma_{\mathrm{eff}}(\omega,\Gamma)
= \gamma_{0}\sqrt{1 + (\omega/\Gamma)^{2}}.
\end{equation}
For $\Gamma \gg |\omega|$, the Markovian limit is recovered. For
$\Gamma \sim |\omega|$, frequency-dependent damping causes the EP to broaden into
a near-critical band
\begin{equation}
0.8 \lesssim \chi \lesssim 1.0.
\end{equation}
In long-memory regimes ($\Gamma \ll |\omega|$) the EP becomes a smooth crossover,
though the peak-merger diagnostic remains detectable if instrumental resolution
is sufficient.

\begin{figure}[t]
  \centering
  \includegraphics[width=0.8\textwidth]{fig3_band_rg.pdf}
  \caption{Left: schematic representation of the near-critical adaptive window
  in the SymC ratio, $0.8 \lesssim \chi \lesssim 1.0$, arising from finite-memory
  broadening in non-Markovian environments. Right: illustrative renormalization-group
  flow of $\chi$ in a toy $\phi^{4}$ model at weak coupling ($\lambda \sim 0.1$),
  showing that $\chi$ drifts by less than $1\%$ across several decades in energy
  scale. Together these panels support the picture of $\chi = 1$ as a structurally
  stable, near-fixed manifold that operationally appears as a narrow band.}
  \label{fig:band-rg}
\end{figure}

\section{Experimental Protocol and Quantitative Falsifier}
Circuit QED provides tunable $\gamma$ via Purcell coupling:
\begin{equation}
\gamma \approx \kappa_{\mathrm{ext}} + \kappa_{\mathrm{int}}.
\end{equation}
A minimal falsification protocol proceeds as follows:
\begin{enumerate}
\item Tune $\chi$ from $0.5$ to $1.5$ by adjusting the external coupling.
\item Measure $S_{11}(\Omega)$ or transmission as a function of frequency.
\item Fit the resulting spectra to one-peak versus two-peak models.
\end{enumerate}
Failure to observe peak merger within $0.8 \leq \chi \leq 1.0$ with resolution
$\delta\Omega \lesssim 0.01\omega$ rules out SymC for that system.

\begin{figure}[t]
  \centering
  \includegraphics[width=0.8\textwidth]{fig4_circuitqed_falsifier.pdf}
  \caption{Conceptual circuit-QED implementation of the SymC falsifier.
  A superconducting resonator of frequency $\omega$ is coupled to an external
  transmission line with tunable coupling $\kappa_{\mathrm{ext}}$, giving a total
  damping rate $\gamma \simeq \kappa_{\mathrm{ext}} + \kappa_{\mathrm{int}}$.
  Sweeping $\kappa_{\mathrm{ext}}$ tunes the SymC ratio $\chi = \gamma/(2|\omega|)$
  from underdamped to overdamped. The predicted doublet-to-singlet spectral
  transition near $\chi \approx 1$ provides a decisive test; failure to observe
  a peak merger in the window $0.8 \le \chi \le 1.0$, at sufficient spectral
  resolution, falsifies the SymC boundary claim for this platform.}
  \label{fig:cqed-falsifier}
\end{figure}

\section{Decoherence and Temperature Dependence}
Lindblad channels that produce $\gamma$ also cause coherence decay. For $\chi < 1$,
oscillations retain partial phase coherence. For $\chi > 1$, coherence collapses
before a full oscillation. At $\chi = 1$, both decay rates coincide. Thermal baths
raise $\gamma(T)$, shifting systems along the SymC axis. The SymC classification
applies at the realized $\gamma(T)$.

\section{Discussion}
SymC provides a structurally well-defined and falsifiable boundary for open QFT
dynamics. Its stability under inheritance, weak RG drift, and finite memory effects
establishes the EP as a structural feature of open quantum fields. The framework
unifies dynamical stability, thermodynamic efficiency, and experimental accessibility.
The appearance of this QFT boundary across scales is a structural consequence of
the SymC framework rather than an analogy.

\begin{thebibliography}{99}

\bibitem{BreuerPetruccione}
H.-P.~Breuer and F.~Petruccione,
\textit{The Theory of Open Quantum Systems}
(Oxford University Press, 2002).

\bibitem{GKS}
V.~Gorini, A.~Kossakowski, and E.~C.~G.~Sudarshan,
``Completely Positive Dynamical Semigroups of N-Level Systems,''
J.~Math.~Phys.~\textbf{17}, 821 (1976).

\bibitem{Lindblad}
G.~Lindblad,
``On the Generators of Quantum Dynamical Semigroups,''
Commun.~Math.~Phys.~\textbf{48}, 119 (1976).

\bibitem{CaldeiraLeggett}
A.~O.~Caldeira and A.~J.~Leggett,
``Quantum Tunneling in a Dissipative System,''
Ann.~Phys.~\textbf{149}, 374 (1983).

\bibitem{Weiss}
U.~Weiss,
\textit{Quantum Dissipative Systems}
(World Scientific, 1999).

\bibitem{BenderBoettcher}
C.~M.~Bender and S.~Boettcher,
``Real Spectra in Non-Hermitian Hamiltonians Having PT Symmetry,''
Phys.~Rev.~Lett.~\textbf{80}, 5243 (1998).

\bibitem{HeissReview}
W.~D.~Heiss,
``The Physics of Exceptional Points,''
J.~Phys.~A: Math.~Theor.~\textbf{45}, 444016 (2012).

\bibitem{AshidaUeda}
Y.~Ashida, Z.~Gong, and M.~Ueda,
``Non-Hermitian Physics,''
Adv.~Phys.~\textbf{69}, 3 (2020).

\bibitem{ClerkRMP}
A.~A.~Clerk, M.~H.~Devoret, S.~M.~Girvin, F.~Marquardt, and R.~J.~Schoelkopf,
``Introduction to Quantum Noise, Measurement, and Amplification,''
Rev.~Mod.~Phys.~\textbf{82}, 1155 (2010).

\bibitem{NoriOpenQFT}
I.~Nakamura and F.~Nori,
``Open Quantum Field Theories and Dissipative Bosonic Fields,''
Phys.~Rev.~A~\textbf{98}, 012105 (2018).

\bibitem{Keldysh1964}
L.~V.~Keldysh,
``Diagram Technique for Nonequilibrium Processes,''
Sov.~Phys.~JETP~\textbf{20}, 1018 (1965).

\bibitem{Sieberer2016}
L.~M.~Sieberer, M.~Buchhold, and S.~Diehl,
``Keldysh Field Theory for Driven Open Quantum Systems,''
Rep.~Prog.~Phys.~\textbf{79}, 096001 (2016).

\end{thebibliography}

\end{document}

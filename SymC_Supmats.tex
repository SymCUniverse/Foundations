\documentclass[11pt]{article}

\usepackage[top=0.50in,bottom=0.75in,left=0.75in,right=0.75in]{geometry}
\usepackage{amsmath,amssymb,amsthm,amsfonts}
\usepackage{hyperref}
\usepackage{graphicx}
\usepackage{cite}

\title{Supplementary Materials for SymC v3.1:\\
Loop-Induced Operators, QCD Substrate Damping, and Lattice Protocols}

\author{Nate Christensen\\
SymC Universe Project\\
NateChristensen@SymCUniverse.com}

\date{1 February 2026\\v3}

\begin{document}

\maketitle

\section*{Overview}

These supplementary materials provide the technical derivations supporting the main text of SymC v3.1. The goals are:

\begin{itemize}
\item to construct the gauge-invariant higher-dimensional operators that couple fermions to stabilized substrates,
\item to derive the loop-induced coefficients relevant for the electron mass,
\item to define the QCD substrate damping rate $\Gamma_{\rm QCD}$ precisely in terms of spectral functions,
\item to present a lattice QCD protocol for extracting $\Gamma_{\rm QCD}$, $\langle G^2\rangle$, and the effective operator coefficients,
\item to provide numerical benchmarks for the substrate-induced electron mass relation.
\end{itemize}

All notation matches the main manuscript.

\section{Appendix A: Gauge-Invariant EFT Construction}

\subsection{A.1 Operator basis}

The lowest-dimension gauge-invariant operator coupling the lepton bilinear to the gluonic scalar operator $G^2$ is dimension 8 before electroweak symmetry breaking:

\begin{equation}
\mathcal{O}_{eG^2} = \varepsilon^{ij} (\bar{L}_i H_j e_R) G^2.
\end{equation}

The mass dimensions are:
\[
\dim(\bar{L} H e_R) = 4, \qquad \dim(G^2) = 4,
\]
so $\dim(\mathcal{O}_{eG^2}) = 8$.

The corresponding EFT Lagrangian term is

\begin{equation}
\mathcal{L}_{\rm eff} \supset \frac{c_e}{\Lambda^4} \varepsilon^{ij} (\bar{L}_i H_j e_R) G^2 + \text{h.c.},
\end{equation}

where $\Lambda$ is the matching scale (electroweak or heavier), $c_e$ is a dimensionless loop-induced coefficient, and $\varepsilon^{ij}$ is the antisymmetric ${\rm SU}(2)_L$ tensor. This operator is the complete leading-order term in the EFT expansion; higher-dimension operators are suppressed by additional powers of $1/\Lambda$.

\subsection{A.2 After electroweak symmetry breaking}

When $H$ acquires a vacuum expectation value $v/\sqrt{2}$, the operator reduces to a dimension-7 scalar coupling:

\begin{equation}
\mathcal{L}_{\rm eff} \supset 
\frac{\tilde{c}_e}{\Lambda^3} (\bar{e} e) G^2,
\end{equation}

where $\tilde{c}_e = c_e v/(\sqrt{2}\Lambda)$.

This is the leading gauge-invariant operator that couples the electron bilinear to the stabilized QCD substrate.

\subsection{A.3 Matching to the electron mass}

The induced electron mass is

\begin{equation}
m_e = \kappa_e \Lambda_{\rm QCD},
\end{equation}

with

\begin{equation}
\kappa_e = 
\frac{\tilde{c}_e}{\Lambda^3}
\frac{\langle G^2\rangle}{\Lambda_{\rm QCD}}.
\end{equation}

This relation is used in the main text to match $\kappa_e$ to the observed Yukawa coupling.

\subsection{A.4 Renormalization scheme dependence}

The coefficient $\tilde{c}_e$ and the condensate $\langle G^2\rangle$ are both scheme-dependent \cite{SVZ,Narison}. The physical mass $m_e$ is scheme-independent, which requires that the scheme dependence of $\tilde{c}_e$ and $\langle G^2\rangle$ cancel in the product. Standard choices include the $\overline{\rm MS}$ scheme at scale $\mu = 2$ GeV for $\langle G^2\rangle$. The matching calculation relating $\kappa_e$ to $y_e$ must be performed in a consistent scheme, with uncertainties estimated by varying $\mu$ over a reasonable range.

\section{Appendix B: Loop-Induced Coefficients}

\subsection{B.1 Origin of $c_e$}

The operator $\varepsilon^{ij} (\bar{L}_i H_j e_R) G^2$ is generated by two-loop electroweak diagrams involving:

\begin{itemize}
\item a Higgs insertion,
\item a lepton line,
\item two gluon legs coupling through a heavy-quark loop.
\end{itemize}

The dominant contribution arises from the top-quark loop due to its large Yukawa coupling.

A parametric estimate is:

\begin{equation}
c_e \sim 
\frac{\alpha_s}{(16\pi^2)^2}
y_e y_t^2.
\end{equation}

\subsection{B.2 Loop estimate interpretation}

The appearance of $y_e$ in the parametric estimate deserves clarification. If one matches within the Standard Model alone, the operator coefficient is necessarily proportional to the SM Yukawa coupling $y_e$. In the SymC framework, the intent is to treat the measured $m_e$ as fixing $\kappa_e$ directly, while a future UV completion would determine $c_e$ without taking $y_e$ as a primitive input. The present calculation shows consistency: given the observed $m_e$ and the QCD condensate, the inferred $\kappa_e$ reproduces the SM value of $y_e$ after radiative matching, without introducing additional parameter tunings.

\subsection{B.3 Matching scale}

The natural matching scale is $\Lambda \sim m_t$ or $\Lambda \sim v$.

We adopt $\Lambda \sim v$ for consistency with the main text. The scale choice introduces logarithmic uncertainties of order $\log(v/m_t) \sim 1$, which are subdominant to the loop suppression.

\subsection{B.4 Numerical estimate}

Using $\alpha_s(m_Z) \sim 0.118$ (evolved to $\mu \sim v$ gives $\alpha_s(v) \sim 0.09$), $y_t \sim 1$, and $y_e \sim 3\times 10^{-6}$:

\begin{equation}
c_e \sim \frac{0.09}{(16\pi^2)^2} \times 3 \times 10^{-6} \times 1 \sim 1.1 \times 10^{-11}.
\end{equation}

After EWSB:

\begin{equation}
\tilde{c}_e \sim \frac{v}{\Lambda} c_e \sim c_e,
\end{equation}

for $\Lambda \sim v$, so $\tilde{c}_e \sim 10^{-11}$.

This estimate is parametric and assumes tree-level matching. A full two-loop calculation with proper renormalization and scheme choices would refine this to within factors of 2 to 3. The uncertainty in $c_e$ propagates to the matching between $\kappa_e$ and $y_e$, but does not affect the prediction $m_e = \kappa_e \Lambda_{\rm QCD}$ itself, which is fixed by data.

\section{Appendix C: QCD Substrate Damping and $\chi_{\rm QCD}$}

\subsection{C.1 Definition of $\Gamma_{\rm QCD}$}

The QCD substrate is modeled as the dominant scalar gluonic mode in the $0^{++}$ channel.

The damping rate is defined as the spectral width:

\begin{equation}
\Gamma_{\rm QCD} = -2 \, {\rm Im} \, \omega_{\rm pole},
\end{equation}

where $\omega_{\rm pole}$ is the pole of the retarded correlator

\begin{equation}
G_R(\omega,\vec{p}=0) = 
\int d^4x \, e^{i\omega t}
\langle [G^2(x), G^2(0)]\rangle.
\end{equation}

Equivalently, in Euclidean space at zero temperature:

\begin{equation}
C(t) = \langle G^2(t) G^2(0)\rangle
\sim A e^{-m_{0^{++}} t}
\end{equation}

defines the pole mass $m_{0^{++}}$.

Near the confinement transition, assuming quasi-particle behavior and weak thermal broadening, the correlator takes the form

\begin{equation}
C(t) \sim A e^{-(m_{0^{++}} + \Gamma_{\rm QCD}/2) t}.
\end{equation}

Deviations from exponential decay (for example, power-law tails or multi-exponential structure) would require modified extraction protocols using spectral-function reconstruction methods such as maximum entropy or Backus-Gilbert \cite{MeyerBulk,AartsReview}.

\subsection{C.2 SymC prediction}

The SymC stability condition requires

\begin{equation}
\chi_{\rm QCD} = 
\frac{\Gamma_{\rm QCD}}{2\Lambda_{\rm QCD}} \approx 1,
\end{equation}

so

\begin{equation}
\Gamma_{\rm QCD} \approx 2\Lambda_{\rm QCD}.
\end{equation}

More precisely, the framework predicts $\Gamma_{\rm QCD} \in [1.5\Lambda_{\rm QCD}, 2.5\Lambda_{\rm QCD}]$, corresponding to $\chi_{\rm QCD} \in [0.75, 1.25]$. Continuum-extrapolated lattice results with $\Gamma_{\rm QCD} < \Lambda_{\rm QCD}$ or $\Gamma_{\rm QCD} > 3\Lambda_{\rm QCD}$ at $>3\sigma$ confidence, after accounting for systematic uncertainties, would challenge this prediction.

\section{Appendix D: Lattice QCD Protocol}

\subsection{D.1 Operators}

We define:

\begin{equation}
O_g = G^2 = G^a_{\mu\nu} G^{a\mu\nu},
\qquad
O_e = \bar{e} e.
\end{equation}

The gluonic operator $O_g$ is constructed using the plaquette or clover-leaf discretization on the lattice. The fermionic operator $O_e$ uses standard Wilson or staggered fermion actions.

\subsection{D.2 Two-point correlators}

The scalar glueball correlator \cite{TeperReview,ChenEtAlGlueball}:

\begin{equation}
C_{gg}(t) = \langle O_g(t) O_g(0)\rangle.
\end{equation}

Extract:

\begin{itemize}
\item $m_{0^{++}}$ (pole mass), via exponential fit to $C_{gg}(t) \sim A e^{-m_{0^{++}} t}$ at large $t$,
\item $\Gamma_{\rm QCD}$ (thermal width), via fits near the confinement transition temperature $T_c$.
\end{itemize}

The extraction of $\Gamma_{\rm QCD}$ requires computing $C_{gg}(t)$ at finite temperature $T \approx T_c$, performing spectral-function reconstruction (for example, using maximum entropy methods \cite{AartsReview}), and identifying the width of the dominant scalar peak. Finite-size effects must be controlled by ensuring $L > 3$ fm, and continuum extrapolation ($a \to 0$) is required to remove lattice artifacts.

\subsection{D.3 Three-point correlators}

To extract the effective coupling:

\begin{equation}
C_{egg}(t,t') = 
\langle O_e(t) O_g(t') O_g(0)\rangle.
\end{equation}

This determines the matrix element:

\begin{equation}
\langle e | G^2 | e \rangle
\sim \frac{C_{egg}}{C_{gg}}.
\end{equation}

The ratio $C_{egg}/C_{gg}$ isolates the coupling strength between the electron and the gluonic substrate, which is related to the effective coefficient $\tilde{c}_e$ via matching calculations.

\subsection{D.4 Matching to $\kappa_e$}

The lattice-extracted quantities determine:

\begin{equation}
\kappa_e = 
\frac{\tilde{c}_e}{\Lambda^3}
\frac{\langle G^2\rangle}{\Lambda_{\rm QCD}}.
\end{equation}

A continuum-extrapolated lattice result differing from $\kappa_e = 2.6 \times 10^{-3}$ by more than combined statistical and systematic uncertainties at $>3\sigma$ confidence, after accounting for scheme choices (for example, $\overline{\rm MS}$ at $\mu = 2$ GeV) and matching uncertainties, would challenge the substrate inheritance mechanism for the electron mass.

\subsection{D.5 Continuum extrapolation and systematics}

The extraction of $\Gamma_{\rm QCD}$ and $\langle G^2\rangle$ requires:

\begin{itemize}
\item \textbf{Continuum limit:} $a \to 0$ with fixed physical volume. Typical lattice spacings range from $a \sim 0.1$ fm (coarse) to $a \sim 0.04$ fm (fine). Extrapolation to $a = 0$ uses fits in $a^2$ or improved actions with reduced discretization errors.

\item \textbf{Thermodynamic limit:} $V \to \infty$ at fixed temperature. Finite-size effects are controlled by ensuring $L > 3$ fm, where $L$ is the spatial extent of the lattice. For correlators at $T \approx T_c$, $L \sim 4$ to 5 fm is typically required.

\item \textbf{Spectral reconstruction from Euclidean correlators:} The retarded correlator $G_R(\omega, \vec{p}=0)$ is reconstructed from the Euclidean correlator $C(t)$ using maximum entropy methods, Backus-Gilbert, or other inversion techniques. These methods are ill-posed and introduce systematic uncertainties, typically of order 20 to 50 percent on spectral widths.

\item \textbf{Temperature protocol:} To access the relevant regime near the confinement transition, simulations should span $T/T_c \in [0.9, 1.2]$, with $T_c \approx 155$ MeV for QCD with physical quark masses.

\item \textbf{Quark mass dependence:} Results should be obtained at physical quark masses (light quark masses corresponding to physical pion mass $m_\pi \approx 140$ MeV). Simulations at heavier pion masses require chiral extrapolation, introducing additional systematic uncertainties.
\end{itemize}

Current lattice QCD technology allows for percent-level precision on masses and condensates in the vacuum, but finite-temperature spectral functions remain challenging. The systematic uncertainty on $\Gamma_{\rm QCD}$ is estimated to be of order 30 to 50 percent with current methods.

\section{Appendix E: Benchmark Region}

Using:
\[
\Lambda_{\rm QCD} \approx 200 \, {\rm MeV},
\qquad
m_e = 0.511 \, {\rm MeV},
\]
we obtain:
\[
\kappa_e \approx 2.6 \times 10^{-3}.
\]

Loop estimates give:
\[
c_e \sim 1 \times 10^{-11},
\qquad
\tilde{c}_e \sim c_e.
\]

The required value of $\kappa_e$ is consistent with:

\begin{itemize}
\item loop suppression ($\alpha_s/(16\pi^2)^2 \sim 10^{-6}$),
\item Yukawa suppression ($y_e y_t^2 \sim 3 \times 10^{-6}$),
\item condensate enhancement ($\langle G^2\rangle/\Lambda_{\rm QCD}^4 \sim 10$ to 100),
\item scale hierarchy $\Lambda_{\rm QCD} \ll v$.
\end{itemize}

The condensate $\langle G^2\rangle$ is known from QCD sum rules and lattice simulations \cite{SVZ,Narison} to be of order $\langle G^2\rangle \sim (0.5 \, {\rm GeV})^4$ in the $\overline{\rm MS}$ scheme at $\mu = 2$ GeV. This translates to $\langle G^2\rangle/\Lambda_{\rm QCD}^4 \sim 40$, providing the necessary enhancement to bridge the gap between the loop-suppressed coefficient $c_e \sim 10^{-11}$ and the observed $\kappa_e \sim 10^{-3}$.

The matching relation
\[
y_e = \kappa_e \sqrt{2} \frac{\Lambda_{\rm QCD}}{v}
\]
then yields
\[
y_e \sim 2.6 \times 10^{-3} \times 1.4 \times \frac{200 \, {\rm MeV}}{246 \, {\rm GeV}} \sim 3 \times 10^{-6},
\]
in agreement with the Standard Model value.

\section{Appendix F: Critical Damping in the Quark Sector}

The stability condition $\chi = \Gamma/(2\Omega) \approx 1$, derived in the main text for the electron coupling to the gluonic condensate, also characterizes three independent systems within the quark sector. These systems differ in scale, symmetry properties, and physical interpretation, yet all exhibit critical damping behavior at characteristic transition points. This appendix presents the technical analysis supporting the quark-sector discussion in Section 9.1 of the main text.

\subsection{F.1 The $\sigma$-Meson: Chiral Condensate Fluctuation}

The spontaneous breaking of chiral symmetry in QCD is associated with a scalar order parameter $\langle\bar{q}q\rangle$. The amplitude fluctuation of this condensate manifests as the $\sigma$-meson (or $f_0(500)$), the lightest scalar-isoscalar resonance. Unlike narrow resonances, the $\sigma$ is notoriously broad, with Particle Data Group values \cite{PDG}:
\begin{equation}
M_\sigma \approx 400\text{--}550 \, {\rm MeV}, \qquad \Gamma_\sigma \approx 400\text{--}700 \, {\rm MeV}.
\end{equation}

The corresponding damping ratio is:
\begin{equation}
\chi_\sigma = \frac{\Gamma_\sigma}{2M_\sigma}.
\end{equation}

Taking central values ($M_\sigma \approx 475$ MeV, $\Gamma_\sigma \approx 550$ MeV) yields $\chi_\sigma \approx 0.58$. Taking upper width limits ($M_\sigma \approx 400$ MeV, $\Gamma_\sigma \approx 700$ MeV) yields $\chi_\sigma \approx 0.88$. Recent dispersive analyses using Roy equations give pole positions around $M_\sigma \sim 440$ to 540 MeV with $\Gamma_\sigma \sim 250$ to 500 MeV \cite{CapriniEtAl2006}, confirming the $\sigma$ lies in the near-critical regime $\chi_\sigma \sim 0.5$ to 0.9.
\begin{equation}
M_\sigma \approx 400\text{--}550 \, {\rm MeV}, \qquad \Gamma_\sigma \approx 400\text{--}700 \, {\rm MeV}.
\end{equation}

The corresponding damping ratio is:
\begin{equation}
\chi_\sigma = \frac{\Gamma_\sigma}{2M_\sigma}.
\end{equation}

Taking central values ($M_\sigma \approx 475$ MeV, $\Gamma_\sigma \approx 550$ MeV) yields $\chi_\sigma \approx 0.58$. Taking upper width limits ($M_\sigma \approx 400$ MeV, $\Gamma_\sigma \approx 700$ MeV) yields $\chi_\sigma \approx 0.88$. Recent dispersive analyses using Roy equations give pole positions around $M_\sigma \sim 440$ to 540 MeV with $\Gamma_\sigma \sim 250$ to 500 MeV \cite{CapriniEtAl2006}, confirming the $\sigma$ lies in the near-critical regime $\chi_\sigma \sim 0.5$ to 0.9.

In all parameter regions consistent with data, the $\sigma$-meson exhibits $\chi \sim \mathcal{O}(1)$, placing it near or at the critical damping boundary. In the SymC framework, this large width is not a defect but a signature: it identifies the chiral condensate as a stabilized substrate optimized for rapid settling ($T_s \to {\rm min}$), distinct from underdamped resonances like the $\rho$ or $\omega$ mesons (which have $\Gamma/M \sim 0.1$ to 0.2) that do not serve as vacuum condensates.

The $\sigma$'s exceptional breadth reflects its role as the "Higgs of QCD," the scalar mode associated with spontaneous chiral symmetry breaking. Just as the electroweak Higgs develops a vacuum expectation value and gives mass to fermions, the chiral condensate $\langle\bar{q}q\rangle$ breaks chiral symmetry and generates constituent quark masses. The SymC framework predicts that such order-parameter fluctuations should exhibit $\chi \approx 1$ to minimize relaxation time during the symmetry-breaking transition.

\subsection{F.2 Dressed Quark Propagator and Confinement}

Dynamical mass generation for light quarks exhibits critical damping in a different manifestation: the absence of a real pole in the quark propagator. In confining theories, an isolated quark is never a stable excitation. The dressed quark propagator has complex poles or a branch cut, indicating the quark is a resonance-like excitation in the strongly coupled vacuum rather than a particle with a well-defined mass shell.

Dyson-Schwinger equation (DSE) studies of the quark propagator in vacuum find complex conjugate pole pairs in the energy plane \cite{RadzhabovEtAl2025,HorakEtAl2023}. For example, Radzhabov et al.\ found that at physical vacuum parameters, the quark propagator poles occur at $M \pm i\Gamma/2$ with $\Gamma$ comparable to $M$. This complex pole structure is the hallmark of critical damping: the quark mode oscillates with frequency $\Omega \sim M$ but is damped on the same timescale $\Gamma^{-1} \sim M^{-1}$.

Recent real-time DSE analyses reconstructed the light quark spectral function and found a broad peak around 0.45 to 0.5 GeV rather than a delta-function pole \cite{HorakEtAl2023}. When meson cloud effects are included in next-to-leading order NJL calculations, the quark spectral function develops a multi-peak structure with a primary constituent mass peak near 0.27 to 0.30 GeV accompanied by substantial continuum strength \cite{MullerEtAl2015}. The primary peak width is not negligible; fits suggest a damping ratio of order unity.

The momentum-dependent quark mass function $M(p^2)$ transitions from the large infrared value (constituent mass $\sim 300$ to 500 MeV for $u,d$ quarks) to the small perturbative current mass at high momenta. This transition region is characterized by a rapid rise in the imaginary part of the quark self-energy. The crossover from confined to perturbative behavior occurs where
\begin{equation}
{\rm Im}\,\Sigma(p^2) \approx 2\,{\rm Re}\,\Omega(p^2),
\end{equation}
which is precisely the SymC condition $\Gamma \approx 2\Omega$. This relation suggests that dynamical chiral symmetry breaking itself is organized around the critical damping boundary.

In essence, the constituent quark mass is not a particle mass but a critically damped collective mode. The quark's typical mass scale ($\sim 300$ MeV for light quarks) is comparable to the virtuality or width of its interactions that confine it. This "no-pole condition" for confinement is a manifestation of $\chi \approx 1$ dynamics in the fundamental fermionic degrees of freedom.

\subsection{F.3 Heavy Quarkonia Dissociation in the Quark-Gluon Plasma}

Heavy quarkonium states (charmonium $c\bar{c}$, bottomonium $b\bar{b}$) provide a third independent test of the $\chi \approx 1$ boundary. In vacuum, these bound states are narrow resonances with $\Gamma \ll M$ (underdamped regime, $\chi \ll 1$). However, in a hot QCD medium near the deconfinement temperature $T_c$, the spectral peaks broaden significantly due to Debye screening and collisional damping.

Lattice QCD studies of charmonium correlators show little change up to $T \approx T_c$, but for $T > T_c$ the correlators deviate, consistent with developing thermal widths in the spectral functions \cite{AartsEtAl2013,MatsufuruEtAl2004}. Reconstructed charmonium spectral functions indicate that the peak width grows as temperature increases. For the $J/\psi$ (the lowest charmonium vector state), the vacuum width is negligible ($\Gamma \sim$ few MeV), but at $T \sim 1.2$ to 1.4$T_c$, substantial broadening is expected:
\begin{equation}
\Omega_{J/\psi} \approx 3.1 \, {\rm GeV}, \qquad \Gamma_{J/\psi}(T \sim T_c) \approx 0.5\text{--}1.0 \, {\rm GeV}.
\end{equation}

At low temperatures, $\chi \ll 1$ (bound state regime). As $T \to T_{\rm dissociation}$, the thermal width increases until the damping dominates the binding frequency. The state "melts" or dissociates into the continuum precisely when
\begin{equation}
\chi_{\rm diss} = \frac{\Gamma(T)}{2\Omega} \to 1.
\end{equation}

Excited charmonium states (such as $\chi_c$ or $\psi'$) dissolve even more rapidly, disappearing at or just above $T_c$. This implies their resonance widths become comparable to the level spacing or binding energy (a scenario equivalent to $\chi \gtrsim 1$). Only the most deeply bound states like $\Upsilon(1S)$ for bottomonium survive well above $T_c$, and even those show noticeable width broadening at higher temperatures.

In potential model language, the quark-antiquark potential above $T_c$ develops a large imaginary part (Debye screening with Landau damping) that rivals the real part (binding potential). This leads to critical damping of quarkonium bound states. The transition from narrow resonances in vacuum to broad, melted states in the plasma coincides with the system crossing the $\chi = 1$ boundary.

Empirically, this is observed as the suppression of excited quarkonia in heavy-ion collisions. The sequential melting pattern (excited states disappear first, ground states persist longer) follows directly from the SymC prediction: states cross $\chi = 1$ when their binding energy equals the thermal damping rate, and more weakly bound states reach this threshold at lower temperatures.

\subsection{F.4 Chiral Restoration and Critical Slowing Down}

The behavior of the $\sigma$-meson near the chiral restoration transition provides additional insight into critical damping dynamics. As QCD approaches the chiral transition near $T_c \sim 155$ MeV (for 2+1 flavors), the scalar and pseudoscalar chiral partners should become degenerate in mass. This temperature-dependent evolution of the $\sigma$ (already identified as a critically damped mode in vacuum) offers a complementary perspective on the $\chi \approx 1$ boundary.

Just below $T_c$, effective models and functional renormalization group (FRG) analyses predict the $\sigma$ meson mass drops toward the pion mass \cite{HatsudaKunihiro2000,TripoltEtAl2014}. This "softening" can reduce the two-pion phase space for $\sigma \to \pi\pi$ decay. Some FRG results suggest the $\sigma$ spectral function develops a low-mass peak near $T_c$ that is narrower than the vacuum resonance, indicating the system approaches a regime where $\chi$ decreases as $\Omega \to 0$ faster than $\Gamma \to 0$.

However, above $T_c$ in the quark-gluon plasma, both scalar and pseudoscalar correlation functions lose distinct resonances. Lattice simulations at $T > T_c$ show the scalar and pseudoscalar screening masses become degenerate and dominated by quark-antiquark continuum excitations. In this regime, any would-be $\sigma$ or $\pi$ mode is heavily overdamped by Landau damping: essentially no quasi-particle peak is resolvable, and spectral functions are smooth and broad, indicating $\chi \gtrsim 1$.

Near the crossover, the system's long-wavelength sigma modes behave as critical fluctuations with large correlation length but slow relaxation (critical damping in the dynamic critical phenomena sense). The dynamical critical exponent for the chiral transition governs the scaling of the relaxation rate of the order parameter. At the critical point, $\Omega \to 0$ (massless mode) while $\Gamma \to 0$ in such a way that $\Gamma/\Omega$ approaches a constant of order unity, characteristic of Model H or Model B dynamics in dynamic universality classes.

\subsection{F.5 Explicit Predictions}

The quark-sector analysis leads to three concrete, falsifiable predictions:

\textbf{(1) $\sigma$-meson stability:} Future dispersive analyses of $\pi\pi$ scattering using Roy equations or lattice QCD determinations of the scalar isoscalar spectral function should continue to find $\Gamma_\sigma/M_\sigma \sim \mathcal{O}(1)$. A sharp, narrow $\sigma$ resonance with $\Gamma_\sigma \ll M_\sigma$ would challenge the SymC identification of the chiral condensate as a critically damped substrate.

\textbf{(2) Quark propagator spectral functions:} Lattice QCD calculations of the light quark spectral function should show ${\rm Im}\,\Sigma(p^2) \sim 2\,{\rm Re}\,\Omega(p^2)$ in the momentum region where the constituent mass transitions to perturbative behavior (typically $p^2 \sim 1$ to 4 GeV$^2$). This can be tested through spectral reconstruction from Euclidean propagators using maximum entropy methods or other inversion techniques. Systematic deviations from this relation across multiple lattice actions and continuum extrapolations would constrain the SymC picture of confinement as a critically damped phenomenon.

\textbf{(3) Quarkonia melting temperatures:} The sequential dissociation pattern of heavy quarkonia in the quark-gluon plasma should align with the condition $\chi(T) = \Gamma_{\rm thermal}(T)/(2\Omega_{\rm binding}) \approx 1$. For charmonium, the $J/\psi$ should melt when $\Gamma_{\rm thermal} \sim 0.6$ to 1.2 GeV (corresponding to $\chi \in [0.8, 1.2]$ for $\Omega_{J/\psi} \approx 3.1$ GeV), while excited states ($\chi_c$, $\psi'$) should dissolve at lower temperatures where their smaller binding energies first satisfy the critical damping condition. Heavy-ion collision data and lattice QCD thermal spectral functions provide independent tests of this prediction.

These predictions are scheme-independent in the sense that the critical damping condition $\chi \approx 1$ is a ratio of physical observables (widths and frequencies) rather than a renormalization-scale-dependent quantity. The numerical windows ($\chi \in [0.8, 1.2]$ for substrates, $\chi \in [0.7, 1.3]$ for falsification thresholds) are specified to account for extraction uncertainties and definition choices, as discussed in Section 8 of the main text.

\subsection{F.6 Connection to Class I Parameters}

The quark-sector $\chi \approx 1$ dynamics provides a foundation for extending substrate inheritance to quark masses and CKM mixing (Class I parameters in Section 6.1 of the main text). If the chiral condensate $\langle\bar{q}q\rangle$, the gluonic condensate $\langle G^2\rangle$, and potentially other scalar substrates all satisfy $\chi \approx 1$, then quark masses can arise from multi-substrate couplings analogous to the electron mass mechanism:
\begin{equation}
m_q = \sum_k \kappa_{q,k} \Lambda_k,
\end{equation}
where $\Lambda_k$ are the characteristic scales of the relevant substrates and $\kappa_{q,k}$ are substrate-coupling coefficients determined by loop-induced operators and spectral overlaps.

The CKM mixing matrix would then arise from the overlap structure of these multi-substrate couplings. In the schematic form of Eq.~(21) in the main text,
\begin{equation}
V_{ij} \sim \mathcal{F}(\kappa_i, \kappa_j, \kappa_k, \ldots),
\end{equation}
the functional $\mathcal{F}$ encodes how different quark flavors couple to different combinations of substrates. The hierarchical pattern of quark masses ($m_u \ll m_c \ll m_t$, etc.) would reflect differences in substrate-coupling coefficients and dominant substrate scales, while the near-unitarity and small off-diagonal elements of the CKM matrix would emerge from the partial alignment of substrate coupling patterns across generations.

Explicit construction of these multi-substrate operators and calculation of the mixing functional $\mathcal{F}$ requires a detailed analysis of flavor-changing gauge-invariant operators in the EFT framework, analogous to the dimension-8 operator constructed in Appendix A for the electron. This is deferred to future work, but the quark-sector $\chi \approx 1$ structure demonstrated here provides the necessary substrate stability foundation for such an extension.

\subsection{F.7 Interpretation and Implications}

The three primary quark-sector systems analyzed above (F.1-F.3) are physically distinct:
\begin{itemize}
\item The $\sigma$-meson is a vacuum scalar resonance associated with the chiral condensate (mass $\sim 500$ MeV).
\item The dressed quark propagator describes the fundamental fermion excitation with dynamical mass generation (constituent mass $\sim 300$ to 500 MeV).
\item Heavy quarkonia are bound states of heavy quark-antiquark pairs (masses $\sim$ few GeV).
\end{itemize}

The chiral restoration dynamics discussed in F.4 represent the temperature-dependent evolution of the $\sigma$ system rather than an independent fourth system. Together, these analyses show that critical damping governs the $\sigma$-meson in both vacuum and finite-temperature environments.

Yet all three independently exhibit the same structural feature:
\begin{equation}
\chi = \frac{\Gamma}{2\Omega} \sim \mathcal{O}(1)
\end{equation}
at the point where the substrate becomes dynamically stabilized or undergoes a phase transition.

This parallel behavior across physically distinct systems spanning six orders of magnitude in mass scale (from $\sim 0.5$ GeV for the $\sigma$ to $\sim 3$ GeV for $J/\psi$) is consistent with the SymC framework in two ways:

\textbf{(1) Cross-scale consistency:} The $\chi \approx 1$ boundary is not tied to a specific operator, particle, or energy scale within QCD. It appears across independent substrates spanning light quarks, heavy quarks, and vacuum condensates.

\textbf{(2) Fermion mass mechanism:} If multiple QCD substrates exhibit $\chi \approx 1$ at their respective characteristic scales, then fermion masses derived from these substrates via the substrate inheritance mechanism (Section 5) naturally inherit similar critical stability structure. This supports the proposal that Standard Model fermion parameters arise from substrate couplings rather than arbitrary Yukawa inputs.

The quark sector thus provides a consistency check: the same $\chi$ boundary condition that explains the electron mass also characterizes quark mass generation, chiral symmetry breaking, confinement, and deconfinement transitions. Whether this pattern extends beyond QCD to other sectors of the Standard Model or to other symmetry-breaking scales remains an open question for future investigation.

\begin{thebibliography}{99}

% Note: This bibliography supplements the main paper references [1]–[27]
% and continues numbering from [28] onwards for clarity when reading both documents together.

% Open quantum systems / GKSL / damping
\bibitem{Davies}
Davies, E. B. (1976).
\emph{Quantum Theory of Open Systems}.
Academic Press.

\bibitem{RivasHuelga}
Rivas, A., \& Huelga, S. F. (2012).
\emph{Open Quantum Systems: An Introduction}.
Springer.

\bibitem{WisemanMilburn}
Wiseman, H. M., \& Milburn, G. J. (2009).
\emph{Quantum Measurement and Control}.
Cambridge University Press.

% Control / settling time / critical damping
\bibitem{Ogata}
Ogata, K.
\emph{Modern Control Engineering}.
Prentice Hall.

\bibitem{FranklinPowellEmami}
Franklin, G. F., Powell, J. D., \& Emami-Naeini, A.
\emph{Feedback Control of Dynamic Systems}.
Pearson.

% Exceptional points / non-Hermitian
\bibitem{Kato}
Kato, T. (1966).
\emph{Perturbation Theory for Linear Operators}.
Springer.

\bibitem{Rotter}
Rotter, I. (2009).
A non-Hermitian Hamilton operator and the physics of open quantum systems.
\emph{Journal of Physics A}, 42, 153001.

\bibitem{Berry}
Berry, M. V. (2004).
Physics of nonhermitian degeneracies.
\emph{Czech. J. Phys.}, 54, 1039.

% EFT / operator language
\bibitem{WeinbergQFT2}
Weinberg, S. (1996).
\emph{The Quantum Theory of Fields, Vol. II}.
Cambridge University Press.

\bibitem{GeorgiEFT}
Georgi, H. (1993).
Effective field theory.
\emph{Annual Review of Nuclear and Particle Science}, 43, 209.

\bibitem{ManoharEFT}
Manohar, A. V. (1997).
Effective field theories.
\emph{Lecture Notes in Physics}, 479, 311.

% Gluon condensate / QCD vacuum
\bibitem{SVZ}
Shifman, M. A., Vainshtein, A. I., \& Zakharov, V. I. (1979).
QCD and resonance physics: theoretical foundations.
\emph{Nuclear Physics B}, 147, 385.

\bibitem{Narison}
Narison, S. (2004).
\emph{QCD as a Theory of Hadrons}.
Cambridge University Press.

% Glueballs / scalar channel
\bibitem{TeperReview}
Teper, M. J. (1998).
Glueball masses and other physical properties of SU(N) gauge theories.
\emph{arXiv:hep-th/9812187}.

\bibitem{AmslerTornqvist}
Amsler, C., \& Tornqvist, N. A. (2004).
Mesons beyond the naive quark model.
\emph{Physics Reports}, 389, 61.

\bibitem{ChenEtAlGlueball}
Chen, Y., et al. (2006).
Glueball spectrum and matrix elements on anisotropic lattices.
\emph{Physical Review D}, 73, 014516.

% Finite-T QCD / spectral functions
\bibitem{MeyerBulk}
Meyer, H. B. (2008).
A calculation of the bulk viscosity in SU(3) gluodynamics.
\emph{Physical Review Letters}, 100, 162001.

\bibitem{AartsReview}
Aarts, G., et al. (2017).
Spectral functions at nonzero momentum in hot QCD.
\emph{Journal of High Energy Physics}, 02, 186.

% Cosmology: structure, growth, GR
\bibitem{Peebles}
Peebles, P. J. E. (1980).
\emph{The Large-Scale Structure of the Universe}.
Princeton University Press.

\bibitem{Dodelson}
Dodelson, S. (2003).
\emph{Modern Cosmology}.
Academic Press.

\bibitem{Linder}
Linder, E. V. (2005).
Cosmic growth history and expansion history.
\emph{Physical Review D}, 72, 043529.

\bibitem{Carroll}
Carroll, S. M. (2004).
\emph{Spacetime and Geometry}.
Addison-Wesley.

% Dark energy / acceleration
\bibitem{Riess1998}
Riess, A. G., et al. (1998).
Observational evidence for an accelerating universe.
\emph{Astronomical Journal}, 116, 1009.

\bibitem{Perlmutter1999}
Perlmutter, S., et al. (1999).
Measurements of $\Omega$ and $\Lambda$ from 42 high-redshift supernovae.
\emph{Astrophysical Journal}, 517, 565.

% Bounce cosmology reviews
\bibitem{BrandenbergerPeter}
Brandenberger, R., \& Peter, P. (2017).
Bouncing cosmologies: progress and problems.
\emph{Foundations of Physics}, 47, 797.

\bibitem{NovelloBergliaffa}
Novello, M., \& Bergliaffa, S. E. P. (2008).
Bouncing cosmologies.
\emph{Physics Reports}, 463, 127.

% Quark sector / sigma meson / chiral dynamics (starting at 28)
\bibitem{PDG}
Particle Data Group (2024).
Review of Particle Physics.
\emph{Progress of Theoretical and Experimental Physics}, 2024, 083C01.

\bibitem{CapriniEtAl2006}
Caprini, I., Colangelo, G., \& Leutwyler, H. (2006).
Mass and width of the lowest resonance in QCD.
\emph{Physical Review Letters}, 96, 132001.

\bibitem{RadzhabovEtAl2025}
Radzhabov, A. E., Volkov, M. K., \& Yudichev, V. L. (2025).
Thermodynamic instabilities in dynamical quark models with contact interaction.
\emph{European Physical Journal C}, 85, 53.

\bibitem{HorakEtAl2023}
Horak, J., Pawlowski, J. M., \& Wink, N. (2023).
Spectral functions in the $\varphi^4$ theory from the spectral DSE.
\emph{SciPost Physics}, 15, 149.

\bibitem{MullerEtAl2015}
Müller, D., Buballa, M., \& Wambach, J. (2015).
Quark propagator and meson correlators at finite temperature and density.
\emph{Physical Review D}, 91, 054004.

\bibitem{AartsEtAl2013}
Aarts, G., et al. (2013).
Charmonium at high temperature in two-flavor QCD.
\emph{Journal of High Energy Physics}, 1312, 064.

\bibitem{MatsufuruEtAl2004}
Matsufuru, H., et al. (2004).
Charmonium correlators at finite temperature.
\emph{Nuclear Physics B Proceedings Supplements}, 129, 499.

\bibitem{HatsudaKunihiro2000}
Hatsuda, T., \& Kunihiro, T. (2000).
Chiral symmetry restoration and the sigma meson.
\emph{arXiv:hep-ph/0010039}.

\bibitem{TripoltEtAl2014}
Tripolt, R.-A., Harterfeld, D., \& Wambach, J. (2014).
Spectral functions from the functional renormalization group.
\emph{Nuclear Physics A}, 928, 149.

\end{thebibliography}

\end{document}